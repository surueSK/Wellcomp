\begin{center}
\textbf{\large Abstract of Bachelor's Thesis Academic Year 2019}

\vspace{6mm}

\textbf{\large Stuguin: The system for internalizing motivation in studying based on user’s motivation type}
\end{center}

\vspace{10mm}


\begin{flushleft}
\textbf{Abstract}\\
\end{flushleft}

Many students feel stressed when studying and 70\% of students don't like learning.
Students who liked learning are characterized by their motivation in learning.

In the self-determination theory, motivation is divided into 6 types based on how internalized the motivation is. 
The more internalized our motivation is, the less stress we feel. 
Each type has a specific cause like punishments, rewards, comparing with others and so on.
And there are many studies that aim to improve motivation by introducing these elements.
The external causes of motivation improve their motivation temporarily, but basically undermine internalizing motivation and degrade their performance.
On the other hand, the property factor of individual’s motivation type fosters intrinsic motivation even if it is external one.

Stuguin proposed in this research is an application for recording study. 
This system judge user's motivation type and toggle displayed features based on the motivation type.

In this research, in order to investigate changes 
in order to investigate changes in motivation in studying by toggling external factors according to the motivation type, we conducted an evaluation experiment over 4 weeks with 15 university students as subjects. 
The subjects were divided into a group displaying functions suitable for the motivational type, a group displaying functions not suitable for the motivational type, and a group displaying no functions.

Through the experiment over 4 weeks, there was no change in motivation in studying.
On the other hand, the subjects who displayed the point function increased the number of study records. 
It means that this system may have improved the motivation for using this application.
In addition, the motivation for using the application should be improved first when we aim to improve motivation in studying by the application.

\begin{flushleft}
\textbf{Keywords}\\
\textbf{Motivation in studying, Mobile learning, Well-being Computing}
\end{flushleft}

\begin{flushright}
\textbf{Keio University Faculty of Environment and Information Studies}\\
\textbf{Satsuki Hashiba}\\
\end{flushright}
\begin{center}
\textbf{\Large 卒業論文要旨 2019年度(令和1年度)}

\vspace{6.18mm}

\textbf{\Large Stuguin: 動機づけタイプに基づいた学習に対する動機づけ内在化システム}
\end{center}

\vspace{10mm}

\begin{flushleft}
\textbf{論文要旨}\\
\end{flushleft}

多くの学生は学習に対して多大なストレスを感じており,7割もの生徒が学習が好きではないと回答した.
対して学習が好きだと回答した生徒らには,学習に対する動機づけに特徴が見られた.

動機づけはその自己決定性に応じて6タイプに分類することができ,それぞれに動機づけが発生する要因が定義されている.
動機づけが内的であるほど,生じるストレスも少なくなる.
罰や報酬,他者との競争などその要因は様々であり,これらの要素を取り入れて動機づけの向上を図る研究は多くなされている.
基本的に外的な動機づけ要因は一時的に動機づけを向上させるものの,内在化を妨げ,パフォーマンスを下げるとされている.
一方で,たとえ外発的に動機づけられた行動であっても,その行動に対する個人の価値の認め方によっては,自己決定性は高くなる.
よって,動機づけタイプに適した外部刺激は,動機づけの向上のみならず内在化も促進できると考えられる.

本研究で提案するStuguinシステムは,学習記録アプリケーションで,ユーザの抱く動機づけタイプを判定し,それに応じて表示される機能が切り替えられるものである.

本研究では,動機づけタイプに応じた外部刺激の切り替えが動機づけの向上及び内在化に有効であるかを調査するため,大学生15名を被験者として4週間に渡る評価実験を行った.
動機づけタイプに適した機能を表示するグループ,動機づけタイプに適していない機能を表示するグループ,機能を表示しないグループに分けた.

4週間の実験を通して学習に対する動機づけの変容は見られず,本システムが学習に対する動機づけに与える影響は確認できなかった.
一方で,ポイント機能を表示した被験者らが学習回数を増加させた変容が見られたことから,本システムがアプリケーション利用に対する動機づけを向上させた可能性が示唆された.
また,アプリケーションを用いて学習の動機づけを向上させようとする場合には,まず先にアプリケーション利用に対する動機づけを向上させる必要があることがわかった.

\begin{flushleft}
\textbf{キーワード}\\
\textbf{学習における動機づけ,Mobile learning,Well-being Computing}

\end{flushleft}

\begin{flushright}
\textbf{慶應義塾大学 環境情報学部}\\
\textbf{羽柴 彩月}
\end{flushright}
\newpage


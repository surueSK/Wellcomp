\chapter{序論}
本章では,はじめに本研究における背景を述べる.
ついで,問題意識および本研究の目的を述べる.
最後に本論文の構成を示す.

\section{背景}
学習は我々の人生を豊かにするものであり,その必要性は誰もが認めるところである.
学習の場として多くの国で義務教育の制度が導入されており,国際連合児童基金UNICEFも平等な教育機会の提供のための活動に力を入れている.
教育を受け学習をする機会は,基本的人権として守られようとしている.

文部科学省の調査~\cite{monkasho}でも8割以上の小・中学性が``勉強は大切である"と回答しているが,一方で,同じ調査の中で``勉強が好きではない"と回答した生徒は約7割にものぼる.
アメリカの大学生を対象に行われた調査~\cite{acha}では,``過去12ヶ月の間に学習に対して強い心理的ストレスを抱いた"という回答が4割を超えた.
学生自殺者の自殺動機に学業不振が挙がるなど~\cite{kourousho},学習に対するストレスは看過できない課題である.

このように学習に対してストレスを抱えていた学生が,そのストレスを抱かなくなった事例がベネッセの継続調査~\cite{benesse}の中で述べられている.
この調査が,中学生の学習に対する意識の変化を追ったところ,1年間で``勉強が嫌い"から``勉強が好き"に回答が変化した生徒が1割ほど存在したという.
該当の生徒は学習時間が増加し,成績も上昇していた.

学習時間が増加した要因の1つとして,学習に対する``動機づけ"に特徴が見られた.
動機づけとは,行動に駆り立て目標に向かわせる内部過程であり,``やる気"という言葉で表現されることも多い.
学習が``嫌いから好き"に変化した生徒は,``新しいことを知るのがうれしい",``自分の希望する高校や大学に進みたい",``友だちに負けたくない"といった動機づけで学習している割合が高かった.
一方で学習が``嫌いのまま"の生徒はその多くが,``先生や親にしかられたくないから"という動機づけで学習していた.
``先生や親にしかられたくないから"という動機づけは,その要因が教師や親からの叱責という外在的なものであり,対して``新しいことを知るのがうれしい"という動機づけは,その要因が知的好奇心という非常に内在的なものである.
ベネッセの調査は,より内在的な動機づけが学習に対するストレスを低減させるとした.

\section{目的}
内在的な動機づけは学習時間を増加させ,これによって成績上昇及び理解度の向上が見込まれる.結果,学習に対する心理的ストレスの軽減につながる.
本研究の目的は,学習に対して内在的な動機づけを付与し,学習者のストレスを低減させることである.

\section{構成}
本論文は,本章を含め全8章からなる.
本章では,本研究における背景と目的を述べた.
第2章では,動機づけに関する関連研究を整理し,動機づけ向上手法について説明する.
第3章では,これまでに開発してきたシステムとその評価結果について説明し,本研究における問題意識について述べる.
第4章では,本研究における要件を述べ,本研究で提案するシステムの概要について説明する.
第5章では,本システムの設計について述べる.
第6章では,本システムの実装について説明する.
第7章では,本システムで得られたデータから動機づけの向上を評価し,考察について述べる.
第8章では,本論文の結論と今後の展望について整理する.
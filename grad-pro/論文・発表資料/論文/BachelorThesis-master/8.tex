\chapter{結論}
本章では,本研究における今後の展望と本論文のまとめを述べる.

\section{今後の展望}
本節では,本研究で提案したStuguinシステムの問題点と解決策を述べ,今後の展望を述べる.

\subsection{アプリケーション利用に対する動機づけ}
本研究はアプリケーションを用いて学習に対する動機づけの向上を狙ったものの,アプリケーション利用に対する動機づけの低さが課題となり,学習に対する動機づけに働きかけることができなかった.
この問題に対する解決策としてまず,被験者群の変更が挙げられる.
今回はドイツ語履修者及び英語履修者に実験参加を依頼し,承諾した者に対して実験を実施したため,学習記録アプリケーションの利用に対する動機づけは検討されていなかった.
アプリケーション利用に対して動機づけを持つ被験者であれば,本システムが学習に対する動機づけに与える影響は大きくなると予想される.
アプリケーションをApp Store~\cite{appStore}やGoogle Play Store~\cite{googlePlay}に公開し,アプリケーションを自らインストールしたユーザのみに被験者を絞ることで,より本システムの効果を高めることができる.
本実験ではランキング機能の集計範囲を全被験者としたが,これに対しグループ分けアルゴリズムを実装し,一般公開を目指したい.

一方で今回のように,アプリケーション利用に対する動機づけが低いユーザも存在する被験者群に対して行う場合には,実験設定の再検討による解決方法が考えられる.
本評価実験では週に一度のアンケート回答以外には自由に利用するようにと指示をしたが,学習時間の記録を強制する実験設定の採用を検討する必要がある.
アプリケーションの利用を強制するために,例えば毎日学習時間の記録を促すプッシュ通知を送信することができる.
より強制力の強い手法として,実験参加に対する謝礼を用意し,謝礼を受け取るためのアプリケーション利用に対する最低限度を設けるといった強制方法も挙げられる.
しかしながら謝礼を用いて利用を強制する手法は,被験者の持つアプリケーション利用に対する動機づけだけでなく,学習に対する動機づけまでをも低下させてしまう可能性がある.
利用を強制する実験設定を採用する場合には,アプリケーション利用に対する動機づけと学習に対する動機づけに関連がないことを示す必要がある.

アプリケーション利用に対する動機づけの低さに起因する問題の解決方法として,アプリケーション利用の必要をなくす方法も挙げられる.
学習時間を自動で検知,計測することができれば,アプリケーションを起動して記入する必要がなくなり,ユーザは受動的にフィードバックを受け取るのみとなる.
フィードバックの閲覧のみであればその負担は非常に少ないため,動機づけ内在化を図るシステムを十分に利用してもらえる可能性が高まる.
また,自動記録機能が実現できれば被験者の全ての学習状況を把握することができるので,学習に対する動機づけへの影響をより正確に評価することができる.
机や筆記具にセンサを取り付けたり,眼鏡型や腕時計型のウェアラブルデバイスを用いるなど,学習の検知方法を探っていきたい.

また,研究目的の再設定も考えられる.
本研究の目的は,学習に対して内在的な動機づけを付与し,学習者のストレスを低減させることであった.
学習に対する動機づけを付与する前に,アプリケーション利用に対する動機づけの付与を目的とすることもできる.
本アプリケーションの利用を動機づけることができれば,それに伴って学習時間が増加し,学習に対する動機づけの内在化に影響を与えることができるのではないだろうか.
学習者の抱くストレス緩和に対して少し間接的なアプローチにはなるが,今回実施した評価実験において,本システムがアプリケーション利用に対する動機づけに有用である可能性が示されているため,その効果は期待できる.
アプリケーション利用の動機づけ向上に焦点を当て,動機づけ判定を行う質問尺度を修正し評価実験の再実施を検討したい.

\subsection{表示機能選定手法の妥当性}
本研究では,動機づけタイプの判定に速水ら~\cite{hayamizu}が作成した学習動機づけ尺度を利用し,この尺度から判定された動機づけタイプのみを用いて表示する機能を決定した.
一方で評価実験では,本システムが表示する機能は学習に対する動機づけではなく,アプリケーション利用に対する動機づけに効果がある可能性が示唆されている.
また,本システムが学習に対する動機づけに働きかけるためには,まず先にアプリケーション利用に対する動機づけを向上させる必要があることもわかった.
本尺度による動機づけタイプとアプリケーションの利用頻度において因果関係は見られず,学習に対する動機づけタイプのみから表示機能を選定するのは適切ではないことが予測される.
したがって今後は,アプリケーションの利用頻度や学習記録回数など,ユーザのアプリケーション利用状況も含めて表示機能を選定する手法を検討したい.

\subsection{被験者が抱く動機づけタイプの偏り}
本実験において,被験者のほとんどが内的調整もしくは同一化的調整で,外的調整と判定されるユーザはおらず,偏りがあった.
学習に対してすでに非常に内在化された動機づけを持っており,故に本システムが及ぼす影響も小さかったものと見られる.
今後は外的調整や取り入れ的調整を抱く被験者を増やして実験を行う必要がある.

\subsection{ランキング機能の動機づけ向上効果}
動機づけを向上させるアプローチの1つとしてランキング機能を実装したが,実験において効果はあまり見られなかった.
ランキングは競争的な要素として動機づけの付与に有用であるとの研究がいくつか存在するものの,今回動機づけに影響を与えなかった原因として,匿名であったことと,被験者らの利用率が低く順位にほとんど変動がなかったことが考えられる.
しかしながらアプリケーションの利用が活発な友人間のグループを作ることは難しく,導入のコストが高い.
したがって今後は,表示する情報を操作するなどして,匿名のランキングであっても動機づけに影響を与えられる手法を調査していきたい.

\section{本論文のまとめ}
本論文では,動機づけを向上させる既存手法を,ユーザの動機づけタイプに合わせて提供することにより,学習に対する動機づけに及ぼす影響を評価するために,Stuguinを提案・実装した.
Stuguinは学習時間を記録するアプリケーションで,ユーザの動機づけタイプに応じて表示される機能が変更されるシステムになっている.
評価実験では総勢15名の被験者に協力してもらい,Stuguinをインストールし,4週間利用してもらった.
動機づけタイプに適した機能が表示されるSame群,動機づけタイプに適していない機能が表示されるWrong群,機能が何も表示されないNone群の3グループに分けた.
4週間の実験を通して学習に対する動機づけの変容は見られず,本システムが学習に対する動機づけに与える影響は確認できなかった.
一方で,ポイント機能を表示した被験者らが学習回数を増加させた変容が見られたことから,本システムがアプリケーション利用に対する動機づけを向上させた可能性が示唆された.
加えて,被験者群間において動機づけの変容や学習記録状況に違いが見られないこと,動機づけタイプが同一化的調整と判定された被験者らが最も多く学習を記録していることから,本アプリケーションを提供すること自体が,同一化的調整による動機づけを抱くユーザに対して影響を与えたと考えられる.
また,アプリケーションを用いて学習の動機づけを向上させようとする場合には,まず先にアプリケーション利用に対する動機づけを向上させる必要があることがわかった.
ランキング機能に関して,動機づけの向上にあたって,匿名性と利用率の低さが課題であることがわかった.
また,目標設定機能が内的調整を抱くユーザに対しても有用である可能性が示された.
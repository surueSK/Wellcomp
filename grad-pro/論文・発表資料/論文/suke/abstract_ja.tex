\begin{center}
\textbf{\Large 卒業論文要旨 2020年度(令和2年度)}

\vspace{6.18mm}

\textbf{\Large ADLogger: 日常生活動作の為のタスク別時間記録システム}
\end{center}

\vspace{10mm}

\begin{flushleft}
\textbf{論文要旨}\\
\end{flushleft}

私たちは時間を用いて生活を続け,時間管理を心がける場面は公私双方様々な場面で存在する.
しかし「逆算の甘さ」によって個人の時間管理に問題がある場合が存在する.

本研究は正確な行動時間把握を目的とするiOSアプリケーションを提案し,
「感覚に依存した見積もりの誤差」「バッファの不備」を回避する事で時間管理行動に対する苦手意識や行動の変化を与える事を目指す.

本研究で使うシステム,「ADLogger」では実測値の記録及び標準偏差の信頼範囲に基づくバッファを考慮した必要時間自動計算機能を搭載している.

本研究では本システム導入前後で見積もり時間や苦手意識の変化が現れるかを調査すべく大学生30名を被験者として約4週間に渡る評価実験を行った.

結果〇〇,考察〇〇(前の文章含め追って推敲する).

\begin{flushleft}
\textbf{キーワード}\\
\textbf{時間知覚(認知),遅刻,行動変容,メタ認知,心理的時間,Well-being Computing}

\end{flushleft}

\begin{flushright}
\textbf{慶應義塾大学 環境情報学部}\\
\textbf{助川 友理}
\end{flushright}
\newpage


\begin{center}
\textbf{\Large 卒業論文要旨 2020年度(令和2年度)}

\vspace{6.18mm}

\textbf{\Large ADLogger: 日常生活動作の為のタスク別時間記録システム}
\end{center}

\vspace{10mm}

\begin{flushleft}
\textbf{論文要旨}\\
\end{flushleft}

私たちは時間を指標に生活を続けており,時間管理を心がける場面は公私双方様々な場面で存在する.
特に朝の支度準備など個人管理の面で時間管理を意識する事は重要である.
一方で,「逆算の甘さ」が原因で時間管理に苦手意識のある人は多い.

本研究は正確な行動時間把握を目的とするiOSアプリケーションを提案し,
時間管理において重要となる「感覚に依存した見積もりの誤差」「バッファの不備」を回避する事で時間管理行動に対する苦手意識や行動の変化を与える事を目指す.
具体的には客観的な実測値及び準備時間において十分な余白時間を入れた時間管理を行う様にする.
本研究で使うシステム,「ADLogger」では実測値の記録及び標準偏差の信頼範囲に基づくバッファを考慮した必要時間自動計算機能を搭載している.
本研究では本システム導入前後で見積もり時間や苦手意識の変化が現れるかを調査すべく大学生30名を被験者として約4週間に渡る評価実験を行った.

結果〇〇,考察〇〇(前の文章含め追って推敲する).

\begin{flushleft}
\textbf{キーワード}\\
\textbf{時間知覚(認知),遅刻,行動変容,メタ認知,セルフマネジメント,Well-being Computing}

\end{flushleft}

\begin{flushright}
\textbf{慶應義塾大学 環境情報学部}\\
\textbf{助川 友理}
\end{flushright}
\newpage


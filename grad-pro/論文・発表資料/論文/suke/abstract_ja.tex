\begin{center}
\textbf{\Large 卒業論文要旨 2020年度(令和2年度)}

\vspace{6.18mm}

\textbf{\Large ADLogger: 日常生活動作の為のタスク別時間記録システム}
\end{center}

\vspace{10mm}

\begin{flushleft}
\textbf{論文要旨}\\
\end{flushleft}
今日私達の生活において,時間管理は欠かせないものになっている.
一方で個人管理としての時間管理はまだ進んでいるとは言い難い.
本研究は正確な行動時間把握を目的とするiOSアプリケーションを提案し,時間管理行動に対する苦手意識や行動の変化を与える事を目的とする.
具体的には実測値の記録及び三点見積もり法を用いた自動計算機能が搭載した.後できちんと書く.




\begin{flushleft}
\textbf{キーワード}\\
\textbf{時間知覚(認知),遅刻,行動変容,メタ認知,心理的時間,Well-being Computing}

\end{flushleft}

\begin{flushright}
\textbf{慶應義塾大学 環境情報学部}\\
\textbf{助川 友理}
\end{flushright}
\newpage


\chapter{評価}
複数タスクにかかる所要時間に対し被験者の予測とADLoggerの予測の差異を比較した上で,
ADLogger導入によって,行動・意識の変容が生じるかどうか評価する.
本章ではまず評価概要を説明し,実験結果を示す.最後に,評価実験から得られた結果をもとに考察を行う.

\section{評価実験の概要}
本研究における評価実験の概要を述べる.はじめに,評価実験を行う目的を説明する.
ついで,評価実験を行う手順について説明する.

\subsection{評価の目的}
本研究では,複数タスクにかかる所要時間の予測に関して被験者の予測とADLoggerの予測の差異を比較する事,
ADLogger導入によって,時間管理に対する苦手意識・行動への変化が起こる事を目的としている.

\subsection{実験手法}
今回の評価実験では,慶應義塾大学の学生男女30名(目標)を対象に再生法\cite{Oguro1961}を参考にして実験を行う.
まず,被験者は自身が保持しているiPhoneにADLoggerをインストールする.
その後,実験で用いたい日常生活動作(3タスク程度,各タスク所要時間10分~20分(or5~10分)程度)及び各タスクの所要時間を被験日前日までに決定し連絡する.
%ーーーーーーーーーーーー!!!ここから要推敲!!!ーーーーーーーーーーーーーーーーーーーー

ーーーー以下メモーーーーーーーーーーーーーーーーーー
被験当日
前日など日を開けてタスク名と総時間を把握
一度に3以上やる
 -実験を知っている人と知らない人で分ける
\UTF{2460}実験準備\UTF{2460}で規定した日時にzoomにて実験者とコンタクトを取る。
\UTF{2461}zoomにてこれから行う行動及び行動毎にかかる時間と予備時間を報告する。
\UTF{2462}報告した行動をアプリで記録し、行動が終了次第zoomに戻り実験者に終了した旨を伝える。
ーーーーーーーーーーーーーーーーーーーーーーーーーー

\subsection{実験環境について}
本人の動機付けが従わない場合や時間に対する自己統制感が少ない場合は時間管理の能力以外で実験に影響を与える可能性がある(松田, 2004).


\subsection{評価手法}
分析手法をいっぱいかくよ!

\subsection{実験終了後アンケート}
実験終了後のアンケートにおいて,被験前後の意識の変化,及びユーザビリティに対する評価を行ってもらう.%ストレスの減少も?
\section{評価結果}
\section{考察}

\section{まとめ}
本章では,評価実験にの概要及び手法についてまとめたた上で,結果・考察を述べた.
次章では,本研究における今後の展望と本論文のまとめを述べる.

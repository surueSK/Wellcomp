\chapter{評価}
複数タスクにかかる所要時間に対し被験者の予測とADLoggerの予測の差異を比較した上で,
ADLogger導入によって,行動・意識の変容が生じるかどうか評価する.
本章ではまず評価概要を説明し,実験結果を示す.最後に,評価実験から得られた結果をもとに考察を行う.

\section{評価実験の概要}
本研究における評価実験の概要を述べる.はじめに,評価実験を行う目的を説明する.
ついで,評価実験を行う手順について説明する.

\subsection{評価の目的}
本研究では,複数タスクにかかる所要時間の予測に関して被験者の予測とADLoggerの予測の差異を比較する事,
ADLogger導入によって,時間管理に対する苦手意識・行動への変化が起こる事を目的としている.

\subsection{実験手法}%ーーーーーーーーーーーー!!!ここから要推敲!!!ーーーーーーーーーーーーーーーーーーーー
今回の評価実験では,再生法\cite{Oguro1961}\cite{Tayama2018}(?)を用いた実験を慶應義塾大学の学生男女30名(目標)に対し実施した.
被験者は実験で用いたい日常生活動作(3タスク程度,各タスク所要時間10分~20分(or5~10分)程度)を決定し,
事前に自身が保持しているiPhoneにADLoggerをインストールしてもらう.
インストール後,ADLoggerを初期設定のままの状態で決定したタスクを時計を見ずにそれぞれ3データ程度記録する.
3データ取得終了後は被験日前日までに各動作の所要総時間総時間,総所要時間を報告する.
予定した被験日時にzoom\cite{zoom}を用いてオンライン会議に参加して貰う.
被験者が入室次第,ADLoggerの設定を変更し,閲覧後実際に行動を行って貰う.
一行動毎にADLoggerに計測を行い,全行動が終了次第zoomに戻り実験者に終了した旨を伝える.

ーーーー以下メモーーーーーーーーーーーーーーーーーー
被験当日
 -実験を知っている人と知らない人で分ける
本人の動機付けが従わない場合や時間に対する自己統制感が少ない場合は時間管理の能力以外で実験に影響を与える可能性がある(松田, 2004).
ーーーーーーーーーーーーーーーーーーーーーーーーーー

\subsection{評価手法}
被験者が3データ記録し,各動作の所要総時間総時間,総所要時間を報告後に以下の様に被験者のグループ分けを行う.
\begin{table}[htb]
\begin{center}
  \begin{tabular}{|l|l|} \hline
   短見積もり群 & 少なくとも1つのタスクを平均より低く申告した群 \\ \hline
   標準見積もり群 & 全てのタスクを平均〜平均+標準偏差*3まで報告した群\\ \hline
   長見積もり群 & 少なくとも1つのタスクを平均+標準偏差*4以上を報告した群 \\ \hline
  \end{tabular}
  \caption{見積もりに関するグループ分け一覧}
  \label{tb:Lakein}
\end{center}
\end{table}
グループ分け後はADLogger閲覧後zoomで行う行動前に見積もりに変化が起きたかを検証する.

ーーーー以下メモーーーーーーーーーーーーーーーーーー
・アプリの方が正しい群と自分で見積もる方が正しい群に分ける
・何をもって「正しい」と言うかは検討の必要有
・仮説検証で効果も見る?
ーーーーーーーーーーーーーーーーーーーーーーーーーー

\subsection{実験終了後アンケート}
実験終了後のアンケートにおいて,被験前後の意識の変化,及びユーザビリティに対する評価を行ってもらう.%ストレスの減少も?

\section{評価結果}

\section{考察}

\section{まとめ}
本章では,評価実験にの概要及び手法についてまとめたた上で,結果・考察を述べた.
次章では,本研究における今後の展望と本論文のまとめを述べる.

\chapter{評価}
再生法\cite{Oguro1961}を用いた実験を行う.


本システムの導入によって,行動・意識の変容が生じるかどうか評価する.
本章ではまず評価概要を説明し,実験結果を示す.最後に,評価実験から得られた結果をもとに考察を行う.

\section{評価実験の概要}
本研究における評価実験の概要を述べる.はじめに,評価実験を行う目的を説明する.
ついで,評価実験を行う手順について説明する.

\subsection{評価の目的}
本研究では,被験者に最適化されたリマインドによって被験者の行動が変化するのかを評価する.
\subsection{実験環境について}
本人の動機付けが従わない場合や時間に対する自己統制感が少ない場合は時間管理の能力以外で実験に影響を与える可能性がある(松田, 2004).

\subsection{実験手法}
ーーーー以下メモーーーーーーーーーーーーーーーーーー
5~10分・10分~20分のタスクをいくつか列挙してもらう
前日など日を開けてタスク名と総時間を把握
一度に3以上やる
 -実験を知っている人と知らない人で分ける
ーーーーーーーーーーーーーーーーーーーーーーーーーー

今回の評価実験では,慶應義塾大学の学生男女X名(目標10名以上)を対象に実験を行う.
被験者は自身が保持しているiPhoneにADLoggerをインストールし,X週間(目標2週間以上)外出日の外出準備時間に使用してもらう,
被験前に2週間分の外出準備の予測を立ててもらう(但し該当日の前日まで編集を可能とする),
最初の1週間目はADLoggerのTODOリスト入力モジュールとストップウォッチモジュールを使用してもらい,日常生活動作別の時間及び総準備時間,更には予測時間とのずれを記録してもらう.2週間目は1週間目のデータを元にリマインドを使ってもらい,リマインドの導入により日常生活動作別の時間の変化や予測時間とのずれに変化が得られるか検証する.
\subsection{評価手法}
分析手法をいっぱいかくよ!

\subsection{実験終了後アンケート}
実験終了後のアンケートにおいて,被験前後の意識の変化,及びユーザビリティに対する評価を行ってもらう.%ストレスの減少も?
\section{評価結果}
\section{考察}
ーーーー以下メモーーーーーーーーーーーーーーーーーーーーーーーーーーーー
・予備時間でこのタスクができます!
・トレーニング用
・スタートをわかりやすく! みたいなアプリの未来性も別途書いて行きたい
ーーーーーーーーーーーーーーーーーーーーーーーーーーーーーーーーーーーー

\section{まとめ}
本章では,評価実験にの概要及び手法についてまとめたた上で,結果・考察を述べた.
次章では,本研究における今後の展望と本論文のまとめを述べる.

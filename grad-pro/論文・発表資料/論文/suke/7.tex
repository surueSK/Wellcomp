\chapter{評価}
複数タスクにかかる所要時間に対し被験者の予測とADLoggerの予測の差異を比較した上で,
ADLogger導入によって,行動・意識の変容が生じるかどうか評価する.
本章ではまず評価概要を説明し,実験結果を示す.最後に,評価実験から得られた結果をもとに考察を行う.

\section{評価実験の概要}
本研究における評価実験の概要を述べる.はじめに,評価実験を行う目的を説明する.
ついで,評価実験を行う手順について説明する.

\subsection{評価の目的}
本研究では,複数タスクにかかる所要時間の予測に関して被験者の予測とADLoggerの予測の差異を比較する事,
ADLogger導入によって,時間管理に対する苦手意識・行動への変化が起こる事を目的としている.

\subsection{実験評価手法}
今回の評価実験では,被験者に時間の長さを教示し,その長さを産生させる時間産生法\cite{Oguro1961}\cite{Tayama2018}を慶應義塾大学の学生男女20名に対し平日と休日(またはそれに準ずる日)にそれぞれ3回(合計6回)に渡って実施した.
始めに,被験者は事前に各自保持しているiPhoneにADLoggerをインストールして貰う.

実験は平日と休日(またはそれに準ずる日)は異なる動き・見積もりを行う可能性を鑑み,3回分の平日と休日をそれぞれ1回ずつ(計6回)実施した.また,○人は追加で○日実施した,
被験日はまず行動前に朝実施する日常生活動に関して行動名,行動毎の必要時間予測,タスクを連続で行った時の総合必要時間予測を申告してもらう.
尚,6回の計測を通じて予測を適宜修正できるものとする.
また,実験期間中はアプリケーションの予測精度を向上させる点及びアプリケーション使用頻度を分析する為,可能な限り被験者は実験用に登録した動作を行う度にアプリケーションに計測して貰った.
その後,web会議ツールであるzoom\cite{zoom}およびADLoggerを用いながら実際に行動して貰い実測値を計測する.
zoomにおいては全ての行動の開始時/終了時に連絡を行い,実験者が総合時間を計測する.
更に被験者はADLoggerを用いて行動毎の時間を計測する.
4回目の計測日においてはADLoggerのADLogおよびタイマーのカウントアップ表示を閲覧できる状態にしながら計測して貰する.
最終日にはプリケーションによる定性的効果やアプリケーションの改善点を知る為にインタビューを行う.インタビューにて質問する項目は付録Aにて記載した.

実測値およびアンケートで申告した見積もりとのずれに関して,被験者の休日と平日のタスク幅,見積もりの傾向によって評価を行った上で,全体の分布を反復測定分散分析(Repeated Measured ANOVA),被験者の精度・正確度の変化の推移の評価によってアプリケーションの効果の評価を行った.更にインタビューにて得られた意見をまとめ記述する.

\section{評価結果}
〜〜ここからは仮置き(【】で括った場合は変化しうる文)〜〜\\
本節では,本システムの導入後時間管理に対し与える影響について,本評価実験で得られた評価結果を示す.
\subsection{被験者の見積もりの傾向}
多くの被験者の見積もりの特徴として,バッファ時間を確保した被験者は【○人】,平均時間は【○分】である事から分かる様に
【「バッファ時間」自体は意図的に加えない組み立て方を行う】被験者が多かった.
追加調査に関しても同様の結果が出た.

\subsection{休日と平日のタスク幅}
インタビューの問3について「差が出る」と答えた被験者と「そうで無い」被験者を対象に標準偏差の比較を行ったところ,
休日と平日によって「差が出る」と答えた被験者は「差が出ない」に比較し,【標準偏差の差が大きかった】.
また,時間管理に対し苦手意識のある被験者は【差が出る被験者が多かった】.
\\
(差が出る/出ないと答えた被験者の比較と苦手意識有/無の被験者の比較の図を挿入予定)
\subsection{精度と正確度の評価}
精度と正確度の分布は以下の様になった.休日と平日によって時間に幅がある被験者は【精度,正確度の結果にも影響があった】.
また,タスク数及び時間によって相関が【あった】.
更に被験者は【実態時間より予測時間の方を多く見積もる】傾向がある事から.
アプリケーションを導入後,下図の様に【精度・正確度の向上が得られた】.
(平日と休日に分けて正確度・精度の遷移を図にまとめて挿入する予定)\\
\subsection{全被験者の反復測定分散分析}
総合時間及び一日あたりの平均のずれの時間を反復測定分散分析(Repeated Measured ANOVA)を用いて分析したところ,以下の様になり,
アプリケーションを導入した2回で有意差を【得られた】.
\\
(有意水準0.05で帰無仮説を行い,結果を表でまとめて挿入する予定)

\subsection{実験終了後インタビューについて}
(上記に言及した事以外のインタビューについて〇〇の様な意見が出された等を書いていく予定)
\section{考察}
(※詳細は実験データを元に書いていきたいと考えております.一先ず
【】の通りに結果が得られた場合に考えられる考察・Limitationを完結に述べます.)

上記結果から【平日/休日の変化が少ない人が時間管理が得意】である事から,規則正しい生活を送っている人ほどルーティンとして確立しており,ルーティンの習熟度が高い事が分かる.
また,【タスク時間・量の少なさとずれ幅の少なさに相関がある】事から,【行うタスクが短く少ない人ほどずれる時間の幅も少なく時間管理がより容易である】事が分かる.
更にアプリケーションの導入ににより時間を可視化しタスク幅を加味したバッファを加える事によって【精度・正確度の向上が得られた】事から
時間の可視化による時間認識の向上とゆとりのあるスケジューリングの導入が【時間管理能力を向上させる】主要因になる事が分かる.

尚,本実験はデータ数・被験者数が少ない事,短期間である事,一人一人のタスク行動が同一ではない事,時間の見積もり自体は個人間でも均一になりにくい事からデータの偏りが考えられる為更なる収集が必要である.
また,データ数の採集を増やし休日/平日を分けた精度・正確度の評価を行う事でより同一条件下での見積もり幅を策定する事が可能である.
特に,被験者の準備時間が短い程見積もりのずれは少ない傾向がある為他の指標で評価する必要がある.
更に,ずれの評価に関しては見積もりの失敗だけでなく複合的な原因が存在する可能性がある為,今後は本評価手法に限定せずに更なる評価実験が必要になると考えられる.
最後にアンケートにおいてはインタビューによる可能性の示唆を元に今後アンケートなどより統計的に評価しなおす意義があると考える.

\section{まとめ}
本章では,評価実験にの概要及び手法についてまとめたた上で,結果・考察を述べた.
次章では,本研究における今後の展望と本論文のまとめを述べる.

\chapter{評価}
複数タスクにかかる所要時間に対し被験者の予測とADLoggerの予測の差異を比較した上で,
ADLogger導入によって,行動・意識の変容が生じるかどうか評価する.
本章ではまず評価概要を説明し,実験結果を示す.最後に,評価実験から得られた結果をもとに考察を行う.

\section{評価実験の概要}
本研究における評価実験の概要を述べる.はじめに,評価実験を行う目的を説明する.
ついで,評価実験を行う手順について説明する.

\subsection{評価の目的}
本研究では,複数タスクにかかる所要時間の予測に関して被験者の予測とADLoggerの予測の差異を比較する事,
ADLogger導入によって,時間管理に対する苦手意識・行動への変化が起こる事を目的としている.

\subsection{実験評価手法}
今回の評価実験では,作成法\cite{Oguro1961}\cite{Tayama2018}を用いた実験を慶應義塾大学の学生男女30名(目標)に対し1ヶ月実施した.
被験者は実験で用いたい日常生活動作を3タスク程度,各タスク所要時間5分から15分程度を決定し,
事前に自身が保持しているiPhoneにADLoggerをインストールしてもらう.

%~~~~~~~~~~~~~~~~~~~~~~~~~~~~~~~ここから!!!~~~~~~~~~~~~~~~~~~~~~~~~~~~~~~~~~~~~~~~~~~~~~~~~~~~~~~

インストール後,ADLoggerを初期設定のままの状態で決定したタスクを時計を見ずにそれぞれ3データ程度記録する.
3データ取得終了後はグループAはバッファを調整しつつADLoggerを閲覧し記録したデータからの自身の経過時間の傾向を閲覧する.
その後被験者は被験日前日までに動作の総所要時間見積もりを予測しアンケートを通じて報告する.
予定した被験日時にzoom\cite{zoom}を用いてオンライン会議に参加して貰う.
被験者が入室次第,実際に行動を行って貰う.
一行動毎にADLoggerに計測を行い,全行動が終了次第zoomに戻り実験者に終了した旨を伝える.

被験当日に行った行動と前日の予測から算出される見積もりのずれをグループ毎に正規分布で表し,t検定(両側検定)を用いて分布の違いを比較する.

また実験終了後はインタビューを行い,被験前後の意識の変化,及びユーザビリティに対する評価を行ってもらう.
インタビューにて聞く項目は下記の通りである.%ストレスの減少も?
\begin{enumerate}
  \item 実験前における時間管理に対する苦手意識の有無
  \item 時間管理に対するストレスの変化
  \item 時間管理に対する苦手意識の変化
  \item 本アプリケーションの見積もり予測に対する主観としての信憑生
  \item 本アプリケーションの良かった点・改善すべき点
  \item 本アプリケーションを可能であれば使い続けたいか否か
  \item ↑の理由
  \item その他感想など
\end{enumerate}

\section{評価結果}

\section{考察}

\section{まとめ}
本章では,評価実験にの概要及び手法についてまとめたた上で,結果・考察を述べた.
次章では,本研究における今後の展望と本論文のまとめを述べる.

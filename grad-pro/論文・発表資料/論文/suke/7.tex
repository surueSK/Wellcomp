\chapter{評価}
複数タスクにかかる所要時間に対し被験者の予測とADLoggerの予測の差異を比較した上で,
ADLogger導入によって,行動・意識の変容が生じるかどうか評価する.
本章ではまず評価概要を説明し,実験結果を示す.最後に,評価実験から得られた結果をもとに考察を行う.

\section{評価実験の概要}
本研究における評価実験の概要を述べる.はじめに,評価実験を行う目的を説明する.
ついで,評価実験を行う手順について説明する.

\subsection{評価の目的}
本研究では,複数タスクにかかる所要時間の予測に関して被験者の予測とADLoggerの予測の差異を比較する事,
ADLogger導入によって,時間管理に対する苦手意識・行動への変化が起こる事を目的としている.

\subsection{実験評価手法}
今回の評価実験では,作成法\cite{Oguro1961}\cite{Tayama2018}を用いた3回の計測実験を慶應義塾大学の学生男女30名(目標)に対し1ヶ月実施した.
始めに,被験者は事前に各自保持しているiPhoneにADLoggerをインストールして貰う.
インストール後,実験で用いたい日常生活動作を3タスク程度,各タスク所要時間5分から15分程度を決定し,タスク名,タスク毎の必要時間予測,タスクを連続で行った時の必要時間予測をADLoggerのアンケート画面から回答する.後日zoom\cite{zoom}を用いながら実際に行動して貰い実測値を計測する.(\UTF{2460})

その後,被験者が実験用に定めた動作3タスク分を初期設定状態のADLoggerを用いて2週間程度各自計測して貰う.
2週間経過後,再びアンケートを回答した上でzoom\cite{zoom}を用いながら実際に行動して貰い計測を行う.(\UTF{2461})

更に,ADLoggerのADLogを用いながら動作3タスク分を更に2週間程度各自計測して貰う.
2週間経過後,再びアンケートを回答した上でzoom\cite{zoom}を用いながら実際に行動して貰い計測を行う.(\UTF{2462})

2回目と3回目のアンケート終了後は,日数が経過しないうちにzoom\cite{zoom}を用いて実験者と面会しインタビューを行う.

評価は\UTF{2460},\UTF{2461},\UTF{2462}時点における見積もりと実測値とのずれの評価,及び心身面の状態変化を比較する.
ずれの評価については全体の分布を反復測定分散分析(Repeated Measured ANOVA)を用いて評価した後,四分位数を用いてグループを分けた時のそれぞれの変化量を確認する.

心身面の状態変化に関しては下記に記すインタビューに基づき評価する.
インタビューにて質問する項目は下記の通りである.%ストレスの減少も?
\begin{enumerate}
  \item 実験前における時間管理に対する苦手意識の有無
  \item 時間管理に対するストレスの変化
  \item 時間管理に対する苦手意識の変化
  \item 本アプリケーションの見積もり予測に対する主観としての信憑生
  \item 本アプリケーションの良かった点・改善すべき点
  \item 本アプリケーションを可能であれば使い続けたいか否か
  \item ↑の理由
  \item その他感想など
\end{enumerate}

\section{評価結果}

\section{考察}

\section{まとめ}
本章では,評価実験にの概要及び手法についてまとめたた上で,結果・考察を述べた.
次章では,本研究における今後の展望と本論文のまとめを述べる.

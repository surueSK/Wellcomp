\chapter{評価}
複数タスクにかかる所要時間に対し被験者の予測とADLoggerの予測の差異を比較した上で,
ADLogger導入によって,行動・意識の変容が生じるかどうか評価する.
本章ではまず評価概要を説明し,実験結果を示す.最後に,評価実験から得られた結果をもとに考察を行う.

\section{評価実験の概要}
本研究における評価実験の概要を述べる.はじめに,評価実験を行う目的を説明する.
ついで,評価実験を行う手順について説明する.

\subsection{評価の目的}
本研究では,複数タスクにかかる所要時間の予測に関して被験者の予測とADLoggerの予測の差異を比較する事,
ADLogger導入によって,時間管理に対する苦手意識・行動への変化が起こる事を目的としている.

\subsection{実験評価手法}
今回の評価実験では,作成法\cite{Oguro1961}\cite{Tayama2018}を用いた実験を慶應義塾大学の学生男女30名(目標)に対し1ヶ月実施した.

被験者は事前に各自保持しているiPhoneにADLoggerをインストールしてもらう.
インストール後,実験で用いたい日常生活動作を3タスク程度,各タスク所要時間5分から15分程度を決定し,タスク名,タスク毎の必要時間予測,タスクを連続で行った時の必要時間予測をADLoggerのアンケート画面から回答する.その後初期設定の状態でに決定したタスクを各々10回以上記録し,アプリケーションの予測精度を向上させる.また,3回以上決定したタスクを連続で行う練習を行う.記録・練習共に終わり次第アンケート画面から再度回答する.

2回目のアンケート終了後,日数が経過しないうちにzoom\cite{zoom}を用いて実験者と面会する.
面会時には初めて設定画面からADLogを解除し今までの自分の行動時間傾向を閲覧し把握した後アンケート画面から再度回答する.
回答後実際に登録したタスクを連続で行い,各タスク及び総タスクの必要時間を計測する.

また実験終了後はインタビューを行い,ADLogger導入後の被験前後の意識の変化,及びユーザビリティに対する評価を行ってもらう.

%~~~~~~~~~~~~~~~~~~~~~~~~~~~~~~~ここから!!!~~~~~~~~~~~~~~~~~~~~~~~~~~~~~~~~~~~~~~~~~~~~~~~~~~~~~~
評価は3つのアンケートの見積もりと最後に計測した実測値とのずれをグループ毎に正規分布で表し,t検定?(両側検定?)を用いて分布の違いを比較する.
インタビューにて聞く項目は下記の通りである.%ストレスの減少も?
\begin{enumerate}
  \item 実験前における時間管理に対する苦手意識の有無
  \item 時間管理に対するストレスの変化
  \item 時間管理に対する苦手意識の変化
  \item 本アプリケーションの見積もり予測に対する主観としての信憑生
  \item 本アプリケーションの良かった点・改善すべき点
  \item 本アプリケーションを可能であれば使い続けたいか否か
  \item ↑の理由
  \item その他感想など
\end{enumerate}

\section{評価結果}

\section{考察}

\section{まとめ}
本章では,評価実験にの概要及び手法についてまとめたた上で,結果・考察を述べた.
次章では,本研究における今後の展望と本論文のまとめを述べる.

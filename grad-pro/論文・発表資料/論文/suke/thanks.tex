\chapter*{謝辞}
本研究を進めるにあたりお世話になった方に謝辞を述べていく.(実験後記載します)
%本研究を進めるにあたって,ご指導頂きました慶應義塾大学環境情報学部教授中澤仁博士に深く感謝いたします.
%また,慶應義塾大学環境情報学部名誉教授徳田英幸博士,慶應義塾大学大学院政策・メディア研究科特任准教授大越匡博士,慶應義塾大学大学院政策・メディア研究科陳寅特任講師,慶應義塾大学大学院政策・メディア研究科研究員伊藤友隆氏,慶應義塾大学大学院政策・メディア研究科研究員柘植晃氏には,本論文の執筆に当たって御助言を賜りました事を深く感謝致します.
%慶應義塾大学中澤研究室の諸先輩方には折りに振れ貴重なご助言を頂きました.
%特に慶慶應義塾大学大学院博士課程佐々木航氏,慶慶應義塾大学大学院博士課程礒川直大氏,ダートマス大学博士課程小渕幹夫氏,慶應義塾大学大学院修士課程栄栄元優作氏,慶應義塾大学大学院修士課程片山晋氏には本論文を執筆するにあたってご指導頂きました.
%また,慶應義塾大学総合政策学部教授藁谷郁美博士には評価実験にあたりご指導,ご協力をいただきました.
%ここに深く感謝の意を表します.

%そして,長時間貴重な時間を割いて実験に協力してくれた櫻井怜氏には深く感謝いたします.
%陰から研究活動を支えて頂いた,松尾さん,遠藤さんに深く感謝申し上げます.
%また,中澤研究室において,柿野優衣氏,川島寛乃氏,谷村朋樹氏,山田佑亮氏,海宝修平氏,鶴岡雅能氏,沼本奨太郎氏,勝又健登氏とは研究活動生活において行動を長く共にし,かけがえのないものとなりました.ここに感謝の意を表します.
%最後に,大学4年間に渡る生活を支えてくれた家族に感謝致します.

\begin{flushright}
\today\\
助川 友理
\end{flushright}

\begin{thebibliography}{100}

\bibitem{drill}
通勤と遅刻の関係 (通勤総合研究所(株式会社ドリル),2017)
\bibitem{kanren2}
金子智栄子,小暮由紀子\\
女子大学生の遅刻に関する研究 遅刻者の状況と意識,並びに性格的特徴と学校適応感について (文京学院大学研究紀要 Vol.7, No.1, pp.193~202, 2005)
\bibitem{Meck2005}
Meck, W. H.(2005)Neuropsychology of timing and time perception. Brain and Cognition, 58, 1-8.
\bibitem{加藤2005}
加藤元一郎(2005) 前頭前野と注意,時間認知.CLINICALNEUROSCIENCE,23(6),632-635.
\bibitem{松田2009a}
時間評価. 松田文子・調枝孝治・甲村和三・神宮英夫・山崎勝之・平伸二(編)『心理的時間—その広くて深いなぞ—』 北大路書房, pp.87-89.
\bibitem{松田堀江一川2011}
BGM 聴取時の心拍数・体温・血圧が時間評価に及ぼす影響. 日本認知科学会 28 回大会論文集, 636-642.
\bibitem{松田一川橘2015}
心拍数が音楽聴取時の時間感覚に与える影響. 日本感性工学会論文集, 14(1), 215-222.
\bibitem{Hoagland1933}
The physiological control of judgments of duration : Evidence for a chemical clock. Journal of General Psychology, 9, 260-287.
\bibitem{Espinosa2003}
Espinosa-Fernandez,L.,Miro,E.,Cano,M.C., Buela-Casal,G. (2003) Age-related changes and gender differences in time estimation. Acta Psychologica, 112, 221-232.
\bibitem{一川2009a}
『大人の時間はなぜ短いのか』 集英社.
\bibitem{加藤宮澤多田 2006}
年を取ると時間の経過を早く感じるようになるか?—高齢者の時間意識—. 日本心理学会第 70 回大会論文発表集, 1151.
\bibitem{和田村田2001}
高齢者の時間感覚に関する研究—高齢者は時間経過をどのように感じるか—. 高齢者問題研究, 17, 79-85.
\bibitem{松田1965}
時間評価の発達I—言語的聴覚刺激のまとまりの効果—. 心理学研究, 36(4),169-177.
\bibitem{松田1967}
時間評価の発達III—標準時間中および再生時間中の音の頻度の効果—. 心理学研究, 37(6), 352-358.
\bibitem{勝浦2007}
勝浦哲夫(2007) 感じ方の色色—光の味覚,時間感覚におよぼす影響. 照明学会誌,91(10),651-654.
\bibitem{藤原狩野1994}
藤原珠江・狩野素朗(1994) VDT 作業での目標設定と即時フィードバックが遂行と時間評価に及ぼす効果. 心理学研究, 65(2), 87-94.
\bibitem{Ono2007}
Ono,F., Kawahara,J(. 2007)Thesubjectivesizeofvisualstimuliaffectstheperceivedduration of their presentation. Perception and Psychophysics, 69(6), 952-957.
\bibitem{Thomas1975}
Thomas, C. E. ,  Cantor, N.(1975) On the duality of simultaneous time and size perception. Perception and Psychophysics, 18(1), 44-48.
\bibitem{島村篠原長山1991}
島村千樹・篠原一光・長山泰久・三浦利章・小川和久(1991) 右折行動の研究(2)—右折所要時間 とその評価—. 日本応用心理学会第 58 回大会発表論文集, 218-219.
\bibitem{篠原2009}
篠原一光(2009) 自動車運転中の時間評価. 松田文子・調枝孝治・甲村和三・神宮英夫・山崎勝之・平伸二(編) 『心理的時間—その広くて深いなぞ—』 北大路書房, pp.303-314.
\bibitem{Burnam1975}
Burnam, M. A., Pennebaker, J.W., Glass, D. C.(1975) Time consciousness, achievement striving, and the Type A coronary-phone behavior pattern. Journal of Abnormal Psychology, 84(1), 76-79.
\bibitem{折原1993}
折原茂樹(1993) Type A と時間評価について—色名呼称版を用いて—. 日本教育心理学会総会発表 論文集, 35, 29.
\bibitem{折原1995}
折原茂樹(1995) 色名呼称盤を用いた時間評価と Type A について. 国士舘大学情報科学センター, 16, 14-21.
\bibitem{新井1985}
新井節男・阪田圭江・内藤純子・福原麻子(1985) 運動中の時間評価とパーソナリティ. 体育の科学, 35(2), 131-134.
\bibitem{Bell1972}
Bell, R. C. (1972) Accurate performance of a time-estimation task in relation to sex, age, and personality variables. Perceptual and Motor Skills, 35, 175-178.
\bibitem{Campos1966}
Campos, P. L. (1966) Relationship between time estimation and retentive personality traits. Perceptual and Motor Skills, 23, 56-62.
\bibitem{Eysenck1959}
Eysenck, J. H. (1959) Personality and the estimation of time. Perceptual and Motor Skills, 9, 405-406.
\bibitem{今井1965}
今井省吾(1965) 時間評価と性格. 日本心理学会第 29 回大会発表論文集, 59.
\bibitem{岩脇1959}
岩脇三良(1959) 時間評価とパーソナリティ特性との関係について. 大脇義一教授在職 35 年記念論文集, 242-250.
\bibitem{加藤1967}
加藤義明(1967) 時間評価と人格的要因. 心理学研究, 38(1), 40-45
\bibitem{Rammsayer1997}
Rammsayer, H. T. (1997) On the relationship between personality and time estimation. Personality and Individual Differences, 23(5), 739-744.
\bibitem{RammsayerRammstedt2000}
Rammsayer, H. T. ,  Rammstedt, B. (2000) Sex-related differences in time estimation : the role of personality. Personality and Individual Differences, 29(2), 301-312.
\bibitem{Wudel1979}
Wudel,P. (1979) Timeestimationandpersonalitydimensions.PerceptualandMotorSkills,48(3), 48.
\bibitem{Bar-Haim2010} 
Bar-Haim, Y., Kerem, A., Lamy, D., Zakay,D.(2010) When time slows down : The influence of threat on time perception in anxiety. Cognition ,  Emotion, 24(2), 255-263.
\bibitem{Hare1963} 
Hare, D. R. (1963) Anxiety, temporal estimation, and rate of counting. Perceptual and Motor Skills, 16, 441-444.
\bibitem{村中坂野2000}
村中泰子・坂野雄二(2000) 時間評価と不安傾向との検討—注意モデルに則して—. 日本行動療法学会第 26 回大会発表論文集, 128-129.
%
\bibitem{一川2008}
知覚体験の時間的特性と心的時間. 辻正二(監修) 『時間学概論』 恒星社厚生閣,pp.119-141.
\bibitem{松田1985}
時間評価とその発達に関するモデル. 心理学評論, 28(4), 597-623.
\bibitem{篠原1996}
時間評価の認知過程—作業記憶の役割—. 大阪大学人間科学部紀要, 22, 71-94.
\bibitem{加藤宮澤多田2006}
年を取ると時間の経過を早く感じるようになるか?—高齢者の時間意識—. 日本心理学会第 70 回大会論文発表集, 1151.
%
\bibitem{酒井2003}
酒井保治.日常生活動作.医学大辞典.第2版,伊藤正男ほか編.医学書院,2003,2095.
\bibitem{train}
定刻発車—日本の鉄道はなぜ世界で最も正確なのか?, 三戸 祐子,新潮文庫, 2005
\bibitem{delay}
『遅刻の誕生』, 本毅彦・栗山茂久,三元社 , 2001
\bibitem{duhouse}
「朝の活動」に関する調査 (株式会社ドゥ・ハウス,2015)
\bibitem{あさとけい}
https://apps.apple.com/jp/app/お出かけ目覚ましアラーム-あさとけい/id561320359
\bibitem{たすくま}
https://apps.apple.com/jp/app/taskuma-taskchute-for-iphone/id896335635
\bibitem{ルーチンタイマー}
https://apps.apple.com/jp/app/ルーチンタイマー/id1455443503

bibitem{kanren7}
Watts FN and Sharrock R:Fear and time estimation.Percept Mot Skills.59: 597-598(1984)
%\begin{comment}
Fortin C. Masse N:Order information in short-term memory and time estimation.Mem and Cogni.t27.54・62. (1999)

%\end{comment}
\end{thebibliography}



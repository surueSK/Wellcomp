\chapter{システム}
本章では,日常生活動作別の時間記録アプリケーション,ADLoggerを提案する.
はじめにADLoggerシステムの概要を述べ,次にADLoggerの特徴を説明する.
そして最後に,ユーザがADLoggerを利用する流れについて述べる.

\section{ADLoggerシステムの概要}
ADLoggerは起床時から出発時刻までの日常生活動作及び経過時間を記録するアプリケーションである.
日常生活動作の内訳及び経過時間を定量的に測定・記録する事が可能であり,実験の客観的な評価が可能となる.
また,本人の行動時間に合わせたバイブレーションによるリマインドが可能である.

\section{ADLoggerシステムの特徴}
本節では,ADLoggerシステムの特徴としてあげられる機能を挙げる.

\subsection{総合時間記録}
ADLoggerはストップウォッチ機能によって開始時刻から外出時刻までの時間を記録する事が出来る.これにより外出準備時間が正確に分かる事が出来る.
\subsection{タスク別ストップウォッチ記録}
ADLoggerは,TODOリスト形式で外出時刻までの日常生活動作を登録する機能がある.リストのカラム毎に隣接したストップウォッチによって日常生活動作毎の時間計測が可能である.
\subsection{バイブレーションによるリマインダー(実装予定)}
得られた総合時間/日常生活動作別行動時間の平均値を算出し,平均値を元に翌日の外出準備時間を予測する.予測データを用いて外出準備時間と半分経過時間,及び1/4,3/4経過時間にバイブレーションによってリマインドを行う.

%\section{ADLoggerシステムの使用方法}
%本アプリケーションを開くと,図~\ref{fig:top}のようなトップ画面が開かれる.

%ユーザはまず,トップ画面下にある``Study"ボタンより,学習を記録する.
%記録には,実際に測定する方法と手動で入力する方法の2種類がある.
%実際に測定する方法を選択した場合には,ストップウォッチのようにして学習時間を測定する.
%測定中の画面を図~\ref{fig:measure4}に示す.

%1件でも学習記録が保存されると,図~\ref{fig:outline4}のようにトップ画面に学習記録の概要が表示されるようになる.
%ユーザは日ごとや週ごとの学習時間を棒グラフで,学習した教科や内容の割合を円グラフで閲覧することができる.
%また,学習を記録した日が色付けされたカレンダーも生成される.

%週に一度,動機づけに関するアンケートがユーザに対して実施される.
%アンケート回答画面を~\ref{fig:questionnaire4}に示す.
%このアンケートによりユーザの動機づけタイプが測定され,その結果に応じてトップ画面に表示される機能が切り替わる.
%ポイント機能では学習時間に応じた獲得ポイントが,ランキング機能は他ユーザも合わせた総学習時間のランキングが表示される.
%目標設定機能の場合は,はじめにユーザはその週の目標学習時間を設定するよう指示される.
%以後は目標に対する達成割合が表示される.
\section{まとめ}
本章では,日常生活動作別行動時間記録及びリマインドを目的としたADLoggerシステムを提案した.
また,ADLoggerシステムの特徴および使用方法を述べた.
次章では,本システムの設計について述べる.
\chapter{システム}
本章では,日常生活動作別の時間記録アプリケーション,ADLoggerを提案する.
はじめにADLoggerシステムの概要を述べ,次にADLoggerの特徴を説明する.
そして最後に,ユーザがADLoggerを利用する流れについて述べる.

\section{ADLoggerシステムの概要}
ADLoggerは行動名別に経過時間を記録するiOSアプリケーションである.
予測算出画面では,行動別の平均時間がリスト形式でカラム毎に出力される.
また,カラムを複数選択する事で複数行動を行う際の必要時間を計算・可視化する事が可能である.

\section{ADLoggerシステムの特徴}
本節では,ADLoggerシステムの特徴としてあげられる機能を挙げる.
\subsection{タスク別ストップウォッチ記録}
簡単な操作で行動名毎に行動時間を記録される.
\subsection{各時間予測}
各行動を下記の計算方法を用いて記録時間の標準的な時間を算出し,リスト形式で行動別に表示する.
\subsection{合計時間の算出}
リストのカラムをタップすると,選択された行動の合計の必要時間を下記の計算手法を元に算出する.

\section{平均時間の算出方法}
\subsection{各時間予測}
行動時間記録の平均時間を算出している.
\[ \bar{x}=\frac{1}{n}\displaystyle\sum_{i=1}^{n}x_{i}\]

\subsection{合計時間の算出}
\[ S_n=\displaystyle\sum_{i=1}^{n} (x_{i}-\bar{x}_{i})+a\]
%これからきちんと書こう。秒単位で計測する分最頻値を出すのが時間がかかる為便宜的に他の手法を取った事も記述しよう。



\section{ADLoggerシステムの使用方法}
本アプリケーションを開くと,ログイン認証後図の様なトップ画面が開かれる.%~\ref{fig:top}

ユーザはまず,トップ画面にある``TIMER"ボタンにより,行動を記録する.
``TIMER"ボタンを押すと,``START"ボタンのあるストップウォッチ画面が現れる.
``START"ボタンを押すとストップウォッチが起動し時間を計測できる.
計測後は``STOP"ボタンを押す.出力されるアラートの中から``終了"ボタンを選択し,タスク名選択画面に移行する.
尚,記録を破棄したい場合はアラートの``Reset"ボタンを,ストップウォッチを止めたくない場合は`計測に戻る"を選択する.

タスク選択画面では行動名がリスト形式で表示されている.一度でも登録された行動名であれば行動名を選択する事で経過時間を保存する事ができる.
新たな行動名であれば``新規追加"ボタンを押し,出力されたアラートに行動名を入力し名前を登録後上記同様に保存する.

一度でも記録時間が保存されると``ADLog"ボタンから行動記録を見る事が可能である.
ユーザは必要に応じてタスクを選択しする事で,複数タスクの合計時間を見る事が可能である.

また,利用規約,実験の説明,アプリの使用方法などはアプリケーションを開いた先にある``HELP"ボタンから確認が可能な様にする.


\section{まとめ}
本章では,日常生活動作別行動時間記録及びリマインドを目的としたADLoggerシステムを提案した.
また,ADLoggerシステムの特徴および使用方法を述べた.
次章では,本システムの設計について述べる.
\begin{center}
\textbf{\large Abstract of Bachelor's Thesis Academic Year 2020}

\vspace{6mm}

\textbf{\large ADLogger:Behavior Modification for ADL Time Management}
\end{center}

\vspace{10mm}


\begin{flushleft}
\textbf{Abstract}\\
\end{flushleft}

There are many situations in which we need to care about time management, both in public and private life such as getting ready in the morning.
However, there are many people who are not good at time management because of lacking extra time as buffer.

In this study, we propose an iOS application  “ADLogger” that aims to estimate the time needed to complete a task.
“ADLogger” is a system that expects how much time you may spend to the task and define the accurate buffer according to your time log data to avoid "errors in estimation depending on the senses" and "inadequate buffers," which are important in time management.
Plus, the system is designed to manage time by including sufficient margin time in objective measurements and preparation time.

In this study, we conducted a four-week evaluation experiment with 20 university students to investigate whether the estimation time and the sense of difficulty change before and after the introduction of this system.

The results suggested that the system reduced the variance between the estimation time and the actual time.
On the other hand, some of the subjects reduced the estimation time in the direction of reducing the margin, and the large individual differences remained as problems.
In the future, it will be necessary to change the functions and optimize the estimation time according to the type of user.

\begin{flushleft}
\textbf{Keywords}\\
\textbf{ Time Perception, Tardiness, Behavior Modification, Metacognition, Self management, Well-being Computing}
\end{flushleft}

\begin{flushright}
\textbf{Keio University Faculty of Environment and Information Studies}\\
\textbf{Yuri Sukegawa}\\
\end{flushright}
\chapter{問題意識}
本章では,本研究における問題意識を洗い出す.
はじめに筆者が本研究に先立ち行った研究について説明し,ついで前章での関連研究も含めた問題意識について述べる.

\section{ADLと遅刻の関連性}
本研究に先立ち,ADLと遅刻の関連性を調べる為,慶應義塾大学生男女12人にインタビューを実施したところ,53\%である8人が「寝坊/ モチベーションに関わらず,時間管理不足により外出時刻に余裕を持って出られなかった経験がある」事に言及しており,各日常動作に対する時間の曖昧性による時間管理の苦手意識があると述べていた.

\section{事前実験について}
朝のADLによる心理的時間のずれを定量的に評価する為に予備実験を実施した(詳細は付録A参照).慶應義塾大学生男女3人に起床後から外出時時刻までの行動別時間に関する予測・実測値の比較及びコンディションとの因果関係を調査した.
\section{問題意識}
心理的時間と物理的時間の解離は確認出来たが,時間に対する注意力と時間管理の関連性の検討がされていない.そこでバイブレーションによるリマインドによって時間に対する注意力の向上を図る.

\section{まとめ}
本章では,筆者が本研究に先立ち行った研究について述べ,問題意識を洗い出した.
次章では,本論文において提案するシステムの要件について述べる.
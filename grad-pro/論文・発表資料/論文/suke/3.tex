\chapter{問題意識}
本章では,本研究における問題意識を洗い出す.
はじめに筆者が本研究に先立ち行った研究について説明し,ついで前章での関連研究も含めた問題意識について述べる.

\section{先行研究からの問題意識}
最適な見積もりを考える手法は大規模なプロジェクト向けのものが多く,日常生活動作向けに個人管理として使えるものか十分な検証はなされていない.
更には,現在存在する時間評価の精度の多くは実験室場面を想定し2分以内の研究が多い為,日常生活動作を評価する30分~60分規模の研究が乏しい.
加えて,システムの提案によって時間の見積もりの誤差が小さくなればどういう効果が現れるかに対して検証する研究はなされていない.
またアプリケーションにおいても,記録されたログの傾向を元に見積もりの精度向上を期待するアプリケーションは乏しい.

\section{逆算の甘さに関して}
先述の通り,遅刻をしてしまう理由の一つでは「時間管理の逆算甘かった」点が挙げられている.%ここの記述を充実させたい.
「逆算の甘さ」が発生する原因としては,「不正確な見積もり時間」及び「余白時間の不十分な確保」が原因であると考えられる(付録1参照).

\section{まとめ}
本章では,筆者が本研究に先立ち行った研究について述べ,問題意識を洗い出した.
次章では,本論文において提案するシステムの要件について述べる.
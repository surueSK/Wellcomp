\chapter{問題意識}

前述した通り,時間の見積もりのズレの原因についての研究や,見積もりの誤差の大きさと時間管理行動尺度の得点差の関連性に対する研究はあるものの(Francis-Smyte \& Robertson, 1999),時間の見積もりの誤差が小さくなれば,適切に時間管理行動を行えるのかを検証する研究は乏しい.アプリケーションにおいても,記録されたログの傾向を元に見積もりの精度向上を期待するアプリケーションは未だにあまりない.

\section{まとめ}
本章では,筆者が本研究に先立ち行った研究について述べ,問題意識を洗い出した.
次章では,本論文において提案するシステムの要件について述べる.
\chapter{問題意識}
本章では,本研究における問題意識を洗い出す.
はじめに筆者が本研究に先立ち行った研究について説明し,ついで前章での関連研究も含めた問題意識について述べる.

\section{個人の逆算の甘さに関して}
背景の項目にて言及した通り,個人の時間管理不備として「時間管理の逆算が甘かった」点が言及されている.
しかし,時間の逆算の甘さは何故発生するのか言及されている研究は乏しい.
そこで本研究に先立ち,タスク別記録アプリケーションを用いて他者の記録を計測する予備実験を行った(付録1参照).
その上でプロジェクト管理同様に見積もりの失敗には「見積もり時間の誤差の大きさ」「バッファの不備」が原因である事が示唆された.

\section{先行研究からの問題意識}
最適な見積もりを考える手法は大規模なプロジェクト向けのものが多く,日常生活動作向けに個人管理として使えるものか十分な検証はなされていない.
更には,現在存在する時間評価の精度の多くは実験室場面を想定し2分以内の研究が多い為,日常生活動作を評価する30分から60分規模の研究が乏しい.
加えて,システムの提案によって時間の見積もりの誤差が小さくなればどういう効果が現れるかに対して検証する研究はなされていない.
またアプリケーションにおいても,記録されたログの傾向を元に見積もりの精度向上を期待するアプリケーションは乏しい.

\section{まとめ}
本章では,筆者が本研究に先立ち行った研究について述べ,問題意識を洗い出した.
次章では,本論文において提案するシステムの要件について述べる.
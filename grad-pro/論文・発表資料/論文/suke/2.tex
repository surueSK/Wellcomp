\chapter{関連研究}
関連研究や用語の定義,先行研究及びアプリケーションの先行事例を示す.

\section{時間の定義}
時間管理の定義は表~\ref{tb:Lakein}に示したLakeinの定義\cite{Lakein1989}をはじめとして様々である.

\begin{table}[htb]
\begin{center}
  \begin{tabular}{|l|l|} \hline
   1 & すべきことを決定する \\ \hline
   2 & 達成するための目標を設定する \\ \hline
   3 & 優先順位を決める \\ \hline
   4 & 取り組む課題のプランニングを作る \\ \hline
  \end{tabular}
  \caption{Lakeinによる時間管理の定義}
  \label{tb:Lakein}
\end{center}
\end{table}

Claessens et al. は,先行研究の定義を俯瞰した上で,時間管理を"目標を達成するために時間を効果的に使用する行動"と定義し時間管理の行動を更に以下の3つに分類した\cite{Claessens2007}(~\ref{tb:Claessens}参照).

\begin{table}[htb]
\begin{center}
  \begin{tabular}{|l|l|} \hline
   時間アセスメント行動(time assessment behavior): \\ ~~~過去,現在,未来の時間を認識し,時間の使い方に関して認識する事 \\ \hline
   プランニング行動(planning behavior): \\  ~~~時間を効率的に使用する事を目的とする事 \\ \hline
   モニタリング行動(monitoring behavior): \\ ~~~行動中における時間の配分のモニタリング・不測の事態へのリスクヘッジ等 \\ \hline
  \end{tabular}
  \caption{Claessens et al. による時間管理の定義}
  \label{tb:Claessens}
\end{center}
\end{table}

\section{先行研究について}
時間管理研究は大きく分けて時間管理がもたらす効果の研究と時間管理能力に関する研究の2種類に分けられる.
前者は更に以下の3つに分類が可能である.(表~\ref{tb:senko})
\begin{table}[htb]
\begin{center}
  \begin{tabular}{|l|l|} \hline
   1 & 時間管理と他の指標の相関関係を調べる研究 \\ \hline
   2 & 時間管理のプロセスモデルの研究 \\ \hline
   3 & 時間管理トレーニングの研究 \\ \hline
  \end{tabular}
  \caption{時間管理がもたらす効果の研究の概要}
  \label{tb:senko}
\end{center}
\end{table}

後者の時間管理能力の研究では必要時間の正確な見積もりの能力に関する研究である.
時間管理能力に関しては主に見積もり時間の精度に関して議論されている.
見積もりの精度は大きく分けて課題に対するもの\footnote{与えられた課題の時間 \cite{Roy2008}や経験の有無\cite{Roy2007}}と被験者の個人差によるもの\footnote{時間評価における知覚時間の歪み\cite{Oguro1961}\cite{Murakami2016}}の2種類存在しているが,
原因として記憶との関連性が考えられている\cite{Roy2005}.

また,特に大掛かりなプロジェクトの際,適切な見積もり時間を計算する為のフレームワークとしてガントチャートやPERT図が使われる場合がある.%要調査!!!

\section{日常生活動作について}
日常生活動作(Activities of Daily Living;ADL)とは,人が日常生活において繰り返す,身の回りの活動や動作のことである.具体的には,身の回りの動作(食事,更衣,整容,排泄,入浴の各動作),移動動作,その他生活関連動作(家事動作,交通機関の利用等)を指す\cite{Sakai2003}.我々は外出準備に平均1時間程度日常生活動作を複数こなしている\cite{duhouse}.本研究では「起床時から外出時刻までに外出準備として行われる日常生活動作」とする.

\section{アプリケーションに関して}
今日日常生活動作を対象にした時間管理に関するiOSアプリケーションが開発されている.例えば,「たすくま」はタスクシュート式\footnote{タスクシュートは大橋悦夫が開発した管理手法であり,1日の仕事を直列に並べ,見積時間を出すと終了時刻を自動予測するシステムを用いて行われる.}のタスク管理アプリである\cite{Taskuma}.「たすくま」はタスク毎の時間を記録すると,予測タスクの自動生成が行われ,日常生活動作を始めとしたルーティンワークの予測が行われる.ルーチンタイマーは複数タスクを「ルーチン」として登録し,設定した所要時間をもとに一つ一つアナウンスされるアプリケーションである\cite{RoutineTimer}.ルーチンタイマーの導入によって対象のルーティンワークの可視化や登録したタスク別終了予定時刻の把握が可能である.

\begin{figure}[ht]
\begin{center}
\begin{tabular}{c}

  	\begin{minipage}[b]{0.5\linewidth}
	\begin{center}
		\fbox{\includegraphics[width=5cm]{images/2/routine1.png}}
		\caption{ルーティンワークカウントダウン}
		\label{fig:goal_alert}
	\end{center}
  	\end{minipage}

  	\begin{minipage}[b]{0.5\linewidth}
	\begin{center}
		\fbox{\includegraphics[width=5cm]{images/2/routine2.png}}
		\caption{ルーティンワークカウントダウン}
		\label{fig:top_goal}
	\end{center}
  	\end{minipage}

\end{tabular}
\end{center}
\end{figure}


\section{まとめ}
本章では,本研究における関連研究を整理し,問題意識を洗い出した.
次章では,筆者が本研究に先立ち行った研究について述べ,問題意識を洗い出す.
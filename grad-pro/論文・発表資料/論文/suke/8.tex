\chapter{結論}
本章では,本研究における今後の展望として改善点及び応用例を述べた後,本論文のまとめを述べる.

\section{改善点}
本節では,実験の反省点と改善策を述べる.

\subsection{ADLogger}
ADLoggerの導入前後で予測時間とのズレが少なくなったと述べたが,これは実態時間が予測時間より長い時だけでなく短い時も同様に発生する.
つまり被験者によっては無意識に時間にゆとりを持っていたにも関わらずADLogger導入によってバッファが短くなってしまう可能性が発生する.
バッファの短縮を阻止し,最適な時間を提案する為には本実験で用いた計算手法の改善やユーザ別もしくはユーザのタイプ別に計算手法を変えていく機能が必要であると考えられる.

また,インタビューを用いて改善すべき点を被験者に伺ったところ,多くの改善案が提案された為,少しずつ改善していく必要がある.
まず短期的なUI改善においては検索機能の搭載やタスク登録後自動的にタイマーが面に戻る機能,ルーティンのオンオフの切り替えの簡易化などが挙げられた.
またタスク登録にて提示される登録タスク候補はサーバに保管されたタスクと同期していない点や自分の設定がADLog画面でしか確認が難しい点が問題点に上がった.
長期的なUI改善としては音声の指示,トラッキングと言った計測の負担を減らす手法が提案された.

\subsection{実験環境}
本実験では被験日には実験者とzoomで話すというプロセスが組み込まれている.
カメラオフ・実験中はマイクオフで参加可能であるとしたものの,実験の度に実験者と会い,
実験終了まで実験者が待っているという環境はそれだけでも必要以上に実験の存在を意識する可能性がある.
実際に被験者の一部はインタビューの際「実験中実験者をを待たせてしまうのは申し訳ないので早めに行動した」と答えた人がいた.
この事からより実態にあった自然なデータを取得する為には実験は実験者と会わずにシステム内で完結させる必要がある.

また,実験は実際に普段行っている朝の支度を再現してもらった為,一人一人行った作業は異なる.
加えて実験した場所はそれぞれの自宅なので環境が異なる場合がある.
例えば実家暮らしだと一人暮らしの被験者とは違い他の家族の行動によって時間をずらす必要が生じる.
特に実験中に引越しした被験者がおり,一部の被験者はルーティンの変化などに影響がある可能性がある.
この事から行うタスクを均質にして同環境下で実験する事併せて行うとより傾向が分かりやすい可能性が高い.

\subsection{評価手法}
今回の実験は短期間かつ少人数で実施した為,アプリケーションの普遍的な効果を本実験・データ数のみで評価することは厳しいと考えられる.
この為被験者人数と日数はより増やした検証を行う必要がある.
特に「逆算が苦手である為時間管理が難しい」人を絞り込み,効果検証を行えばもっと精度の高い評価が得られる可能性がある.

また評価に関して今回の実験はy=axの傾向線を用いて行ったが,そもそも見積もりの予測傾向に関する理想のモデルは線形とは限らない.
予測傾向に関するモデルの先行研究が存在しない為,今後実験を通して最適な傾向モデルを模索していく必要がある.

さらにインタビューで得られた意見・感想元に更なる仮説の検証やインタビューによって得られた因子をもとにアンケートを作成し統計的に評価しなおす意義があると考える.

最後に今回はルーティンに対する慣れの因子がより強く出ていたが,本来は情報処理能力と多次元処理能力と言った「マルチタスク」に関する能力\cite{multitask}や時間感覚\cite{Tayama2018}の個人差による違いも出る可能性がある.
先ほどあげた因子の他にも時間の見積もり自体は個人間でも均一になりにくい事など偏り及び多くの交絡因子が考えられ,時間把握の能力策定には更なる収集が必要である.

\section{応用例}
本節では,今後の展望を応用例と共に述べる.

本アプリケーションと似た機能を応用すれば朝の準備時間に限らず課題など別のタスクへの応用も可能であると考えられる.
さらにアプリケーションのUI改善において加速度センサーなどを用いた動作検知や機械学習を用いた予測精度の向上が期待できる.
また,ユーザ層への理解を深めることによってまだまだ発展途上である時間管理ないしは逆算に関する研究の貢献も期待できる.
\section{本論文のまとめ}
本論文では,タスク毎の時間及び合計時間の見積もり精度を向上させるためにADLoggerを提案・実装した.
ADLoggerはストップウォッチで記録されたでタスク毎の時間を基にユーザの行動時間傾向を予測し必要時間見積もりを簡単に導き出せるアプリケーションである.
評価実験は総勢19名の被験者に協力頂き朝の準備支度を想定した環境でADLoggerを使ってもらい評価実験を行った.
結果見積もりの時間と実態の時間のばらつきを縮小させた事が示唆された.
一方で被験者によっては余裕がある予測時間を削る方向で縮小させてしまったり,個人差の大きさなどが問題点として残った.
今後はユーザのタイプに合わせた機能の変化や見積もりの時間の最適化が求められる.
\chapter{結論}
本章では,本研究における今後の展望と本論文のまとめを述べる.
\section{今後の展望}
本節では,本研究で提案したシステムの問題点と解決策を述べ,今後の展望を述べる.

\subsection{改善点}
今回アプリケーションの介入により被験者によっては予測時間の見積もりが短くなってしまった原因は見積もりの最適化に問題があると考える為,更なる実験でアプリケーションの最適化が必要であると考える.
また,インタビューでは改善すべき点にタイマーを計る過程の効率化や,音声の指示,トラッキングと言った計測の負担を減らす手法やオンオフの差のデータ切り替えのしやすさが提案された為,こうした指摘を元にUI含め改善が必要であると考える.
\subsection{応用例}
本アプリケーションと似た機能を応用すれば朝の準備時間に限らず課題など別のタスクへの応用も可能であると考えられる.
加速度センサーなどを用いた動作検知や機械学習を用いた予測精度の向上が期待できる.
また,ユーザ層への理解を深めることによって時間管理における逆算に関する研究の発展も期待できる.
\section{本論文のまとめ}
本論文では,タスク毎の時間及び合計時間の見積もり精度を向上させるためにADLoggerを提案・実装した.
ADLoggerはストップウォッチで記録されたでタスク毎の時間を基にユーザの行動時間傾向を予測し必要時間見積もりを簡単に導き出せるアプリケーションである.
評価実験は総勢19名の被験者に協力頂き朝の準備支度を想定した環境でADLoggerを使ってもらい評価実験を行った.
結果見積もりの時間と実態の時間のばらつきを縮小させた事が示唆された.
一方で被験者によっては余裕がある予測時間を削る方向で縮小させてしまったり,個人差の大きさなどが問題点として残った.
今後はユーザのタイプに合わせた機能の変化や見積もりの時間の最適化が求められる.
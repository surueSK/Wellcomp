\chapter{序論}
本章では,はじめに本研究における背景を述べる.
ついで,問題意識および本研究の目的を述べる.
最後に本論文の構成を示す.

\section{背景}

私たちは長きに渡って時間を客観的種票として共同生活を続けている\cite{history}.
近年では時間を資源として考え,時間管理を心がける場面は公私双方様々な場面で存在する.
特にプロジェクト管理においては時間管理を非常に重要なものであると位置付けこれまで多くのフレームワークが考案されている.
一方で個人管理としての時間管理はまだ進んでいるとは言い難い.
例えば文京学院大学による遅刻の状況の調査によると授業・友達の待ち合わせ共に「逆算の甘さ」が一因となり遅刻すると考えている人が多数を占める\cite{bunkyo}.

\section{目的}
本研究は行動時間の実測値記録及び必要時間の簡算出機能を搭載したiOSアプリケーションを提案し,時間管理行動に変化を与える事を目的としている.

\section{構成}
本論文は,本章を含め全8章からなる.
本章では,本研究における背景と目的を述べた.
第2章では,関連研究を整理する.
第3章では,これまでに開発してきたシステムとその評価結果について説明し,本研究における問題意識について述べる.
第4章では,本研究における要件を述べ,本研究で提案するシステムの概要について説明する.
第5章では,本システムの設計について述べる.
第6章では,本システムの実装について説明する.
第7章では,本システムで得られたデータから評価を行い,考察について述べる.
第8章では,本論文の結論と今後の展望について整理する.
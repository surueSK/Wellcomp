\chapter{序論}
本章では,はじめに本研究における背景を述べる.
ついで,問題意識および本研究の目的を述べる.
最後に本論文の構成を示す.

\section{背景}
現代は社会活動を送る上で時間を共通の客観的指標の一つとしており,適切な時間管理が求められる機会も多い.しかしながら,「逆算の甘さ」,つまり見積もり時間と実行動のずれが原因となり社会行動に支障をきたす場合がある.例えば文京学院大学による遅刻の状況の調査によると授業・友達の待ち合わせ共に「逆算の甘さ」が一因となり遅刻すると考えている人が78\%程いる\cite{bunkyo}.見積もりの不正確さに対する原因としては記憶バイアスや主観的に感じた感覚時間(心理的時間とも言う)に左右されてしまい,適切な見積もりが出来ない場合があるからと考えられている[].

\section{目的}
本研究では行動時間の実測値を記録するシステムによって見積もり時間を向上させ,時間管理行動の精度を上げる事を目的としている.

\section{構成}
本論文は,本章を含め全7章からなる.
本章では,本研究における背景と目的を述べた.
第2章では,関連研究を整理する.
第3章では,これまでに開発してきたシステムとその評価結果について説明し,本研究における問題意識について述べる.
第4章では,本研究における要件を述べ,本研究で提案するシステムの概要について説明する.
第5章では,本システムの設計について述べる.
第6章では,本システムで得られたデータから評価を行い,考察について述べる.
第7章では,本論文の結論と今後の展望について整理する.
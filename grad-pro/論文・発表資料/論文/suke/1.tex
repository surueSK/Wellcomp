\chapter{序論}
本章では,はじめに本研究における背景を述べる.
ついで,問題意識および本研究の目的を述べる.
最後に本論文の構成を示す.

\section{背景}
私達は時間を共通の客観的指標として時間を遵守しながら社会活動を送っている.しかしながら,事前に開始時刻や日程等が決められている行為への参加に遅れる場合があり,我々はこれを「遅刻」と定義している.殊に日本社会において遅刻は,自らの評価を非常に大きく下げる重大なミスの一つと見なされる傾向にあるが,5.3\%以上の社会人は年間37.6回遅刻すると言う結果が出ている\cite{prtimes}.文京学院大学の研究では, 遅刻の一大要因として「逆算の甘さ」,つまり所要準備時間の予測と実行動のずれを示唆している\cite{kanren2}.所要準備時間の予測と日行動のずれが起きてしまう原因として時間に対する注意不足による心理的時間のずれが考えられる.しかしながら経過時間への注意不足と遅刻の関連性は依然として明らかにされていない.

\section{目的}
本研究では注意力を向上させ心理的時間のずれを解消させることによって外出時刻への遅刻を未然に防ぐ事である.本研究によって外出時刻への遅刻を未然に防ぎ,時間を遵守する事への負担を軽減する事が期待できる.

\section{構成}
本論文は,本章を含め全7章からなる.
本章では,本研究における背景と目的を述べた.
第2章では,関連研究を整理する.
第3章では,これまでに開発してきたシステムとその評価結果について説明し,本研究における問題意識について述べる.
第4章では,本研究における要件を述べ,本研究で提案するシステムの概要について説明する.
第5章では,本システムの設計について述べる.
第6章では,本システムで得られたデータから評価を行い,考察について述べる.
第7章では,本論文の結論と今後の展望について整理する.
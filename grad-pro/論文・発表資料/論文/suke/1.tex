\chapter{序論}
本章では,はじめに本研究における背景を述べる.
ついで,問題意識および本研究の目的を述べる.
最後に本論文の構成を示す.

\section{背景}
時間は出来事や変化を認識するための基礎的な概念であり,長きに渡って身近な存在であり続けている\cite{history}.
時間は学術的対象としてもしばしば扱われており,芸術,哲学,自然科学,心理学など幅広い分野で研究が行われ続けている.

特に産業革命後以降においては,生産性を高めるべく時間管理に関する議論が高まった\cite{Taylor1911}.
プロジェクト管理においてはその意識が顕著であり,「感覚に依存した見積もりの誤差」「バッファの不備」による計画の失敗を非常に懸念する\cite{innopm}.
その為プロジェクト管理に対する計画においては時間管理を非常に重要なものであると位置付ける事が多く,これまで多くのフレームワークが考案されている\cite{EORMS}.

また,時間管理は個人のパフォーマンス\cite{Barling1996}\cite{Britton1991}\cite{Trueman1996}やストレス\cite{Macan1994}などの関連性が考えられており,個人管理の面でも重要視されている.
しかしながら個人管理としての時間管理はプロジェクト管理などと比較してまだ進んでいるとは言い難い.
例えば文京学院大学による遅刻の状況の調査によると授業・友達の待ち合わせ共に「逆算の甘さ」が一因となり遅刻すると考えている人が多数を占める\cite{bunkyo}.

\section{目的}
本研究は個人管理における時間管理の失敗に関する仮説を提案した上で,
行動時間の実測値記録及び必要時間の簡算出機能を搭載したiOSアプリケーションを提案し,苦手意識・行動に対し変化を与える事を目的としている.

\section{構成}
本論文は,本章を含め全8章からなる.
本章では,本研究における背景と目的を述べた.
第2章では,関連研究を整理する.
第3章では,これまでに開発してきたシステムとその評価結果について説明し,本研究における問題意識について述べる.
第4章では,本研究における要件を述べ,本研究で提案するシステムの概要について説明する.
第5章では,本システムの設計について述べる.
第6章では,本システムの実装について説明する.
第7章では,本システムで得られたデータから評価を行い,考察について述べる.
第8章では,本論文の結論と今後の展望について整理する.
\chapter{序論}
本章では,はじめに本研究における背景を述べる.
ついで,問題意識および本研究の目的を述べる.
最後に本論文の構成を示す.

\section{背景}


時間の概念は古来から存在し,現在まで客観的指標の一つとしての役割を現在まで担い続けている\cite{history}.
産業革命後は事業計画をはじめとして目的を達成する為に様々な時間管理手法が考案されてきた.


今日では特にパフォーマンスやストレスなど様々な観点から適切な時間管理が求められている.\cite{Barling1996}\cite{Britton1991}\cite{Burt1994}\cite{Macan1994}.
しかしながら実態として,全ての人が時間管理が完璧にできる訳では無い.
例えば文京学院大学による遅刻の状況の調査によると授業・友達の待ち合わせ共に「逆算の甘さ」が一因となり遅刻すると考えている人が多数を占める.\cite{bunkyo}.
「逆算の甘さ」が発生する原因としては,「不正確な見積もり時間」及び「余白時間の不十分な確保」が原因であると考えられる(付録1参照).


\section{目的}
本研究では行動時間の実測値を記録するシステムによって精度の高い見積もり時間及び十分な余白時間の確保を加味した行動必要時間を提案し,時間管理行動に変化を与える事を目的としている.

\section{構成}
本論文は,本章を含め全7章からなる.
本章では,本研究における背景と目的を述べた.
第2章では,関連研究を整理する.
第3章では,これまでに開発してきたシステムとその評価結果について説明し,本研究における問題意識について述べる.
第4章では,本研究における要件を述べ,本研究で提案するシステムの概要について説明する.
第5章では,本システムの設計について述べる.
第6章では,本システムで得られたデータから評価を行い,考察について述べる.
第7章では,本論文の結論と今後の展望について整理する.
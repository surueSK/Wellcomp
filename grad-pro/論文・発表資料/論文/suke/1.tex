\chapter{序論}
本章では,はじめに本研究における背景を述べる.
ついで,問題意識および本研究の目的を述べる.
最後に本論文の構成を示す.

\section{背景}
時間は出来事や変化を認識するための基礎的な概念である\cite{history}.
時間は学術的対象としても多く扱われており,芸術,哲学,自然科学,心理学など幅広い分野で研究が行われている.

産業革命後以降において,生産性を高めるべく時間を一資源と考え時間管理方法を検討する議論が高まった\cite{Taylor1911}.
プロジェクト管理においては「感覚に依存した見積もりの誤差」「バッファの不備」による計画の失敗を非常に懸念する\cite{innopm}.
こうした失敗を避けるべく現在のプロジェクト管理は主に1940年代から考案された多くの手法やフレームワークを参考にし実際のビジネスに応用する動きが多くみられる\cite{EORMS}.

%ADHDとの関連性も簡潔に書く?
また,近年では生産性だけではなく時間管理能力と個人のパフォーマンスやストレスと言った関連性を考える研究も存在する.
例えば表\ref{tb:dif}の様にそれぞれ定義した質問紙を用いた研究では\cite{Imura2016}時間管理能力の高い人はパフォーマンスの高さ\cite{Barling1996}\cite{Britton1991}\cite{Trueman1996}やストレスの低さ\cite{Macan1994}も伴うという仮説が立てられている\cite{Claessens2007}.

\begin{table}[h]
\begin{center}
\begin{tabular}{|c|c|c|} \hline
Macan(1994) & Britton\&Tesser(1991) & Bond\&Feather(1988)  \\ \hline \hline

\begin{tabular}{c}
時間を管理するための技術
\end{tabular}
 & 
\begin{tabular}{c}
 知的な生産性を最大化すること\\を意図した実践
\end{tabular}
&
\begin{tabular}{c}
 個人が自身の時間を\\どの程度構成し\\目的にかなった様に\\できているかと知覚している程度
\end{tabular}
\\ \hline
Time Management Behavior Scale & Time Management Questionnaire & Time Structure Questionnaire  \\ \hline
\end{tabular}
 \caption{時間管理の定義}
  \label{tb:dif}
\end{center}
\end{table}

時間管理は朝の支度準備など個人管理の側面において時間を管理する機会は日常生活の中でも数多く存在する.
しかし個人による時間管理は完璧であるとは言い難い.
例えば文京学院大学による遅刻の状況の調査によると授業・友達の待ち合わせ共に「逆算の甘さ」によって遅刻すると考えている人が多数を占める\cite{bunkyo}.

\section{目的}
逆算を補助する事で個人の時間管理の精度を向上させ,
時間管理能力やパフォーマンスの向上及び苦手意識やストレス軽減といった精神面に効果が期待できる.
本研究は朝の支度準備における時間管理の失敗に関する仮説を提案・分析した上で,行動時間の実測値記録及び必要時間の簡算出補助を提案し時間管理の精度向上や苦手意識・行動に対し変化を与える事を目的としている.

\section{構成}
本論文は,本章を含め全8章からなる.
本章では,本研究における背景と目的を述べた.
第2章では,関連研究を整理する.
第3章では,これまでに開発してきたシステムとその評価結果について説明し,本研究における問題意識について述べる.
第4章では,本研究における要件を述べ,本研究で提案するシステムの概要について説明する.
第5章では,本システムの設計について述べる.
第6章では,本システムの実装について説明する.
第7章では,本システムで得られたデータから評価を行い,考察について述べる.
第8章では,本論文の結論と今後の展望について整理する.